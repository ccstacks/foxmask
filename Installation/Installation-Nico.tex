\documentclass[a4paper, 11pt]{article}
\usepackage[utf8]{inputenc}
\usepackage[T1]{fontenc}
\usepackage{caption,subcaption,graphicx,tablefootnote,url,listings} %\usepackage[demo]{graphicx}
%\usepackage[frenchb]{babel}
\usepackage[colorlinks=true]{hyperref} %\usepackage[hidelinks]{hyperref}

\newcommand{\mychapter}[2]{
    \setcounter{chapter}{#1}
    \setcounter{section}{0}
    \chapter*{#2}
    \addcontentsline{toc}{chapter}{#2}
}

\hypersetup{
%    colorlinks=true, % make the links colored
    linkcolor=black, % color TOC links in blue
%    urlcolor=red, % color URLs in red
    linktoc=all % 'all' will create links for everything in the TOC
}


\title{Installation of needed dependencies to execute Python Image Classifier}
\author{Eric Devost}
\date{April 2015} 

\begin{document}
\maketitle
\newpage

\section{Python}
Two branches of Python (Python2.x and Python3.x) are actually developped. Python3.x is the new version and the future of the language, however, there is still limited third-party modules and libraries support for it. While Python3 gathers momentum and more libraries are being supported, the use of Python2.x is recommended. The Classifier script has been developped on Python 2.7.9 build on Dec 11 2014 on Arch Linux (GCC 4.9.2). Without modifications, the classifier will not run on Python3.

\subsection{Python dependencies}
Here is the list of imported modules needed to execute the script. Efforts have been made to keep the number of required modules as low as possible and multiplatform.

\begin{lstlisting}
import os
import cv2
import cv2.cv as cv
import numpy as np
import glob
from itertools import groupby
import time
from datetime import datetime,date
import shutil
from gi.repository import GExiv2

\end{lstlisting}

\begin{description}
\item This modules come with a standard Python installation, so you can call it directly with import

\begin{description}
\item [import os]
Used to manipulate path operations with os.path. This is the recommended way for platform independent path manipulations. This module comes with standard Python installation.
\item The next modules come with a standard Python installation, so you can call it directly with import.

\begin{description}
\item[import glob] Used to generate lists of files matching given patern. %Comes with standard Python installation.
\item [from itertools import groupby] Used to create iterators. %Comes with standard Python installation.
\item [import time] Used to manipulate time data. %Comes with standard Python installation.
\item [from datetime import datetime, date] Used to format time data. %Comes with standard Python installation.
\end{description}

\item This three modules will probably have to be installed separately.

\begin{description}
\item [import numpy as np] Numpy libraries (http://numpy.org). A popular module used for miscellaneous math operations. This module may or may not come with your standard Python installation, depending on how you installed Python, but is often provided as a separate package (i.e. To install numpy in Debian or Ubuntu write: "apt install python-numpy", or "yum install python-numpy" respectively).

\item [import cv2 and import cv2.cv as cv] OpenCV2 libraries (http://opencv.org). These are the main libraries used for images manipulations in the script. The versions used for developping the script were OpenCV-2.4.9x and OpenCV2.4.10x. The new development branch of OpenCV (OpenCV3) is still in BETA (as of april 2 2015), so the stable branch (OpenCV2) was used. OpenCV does not come with standard Python installation.

\item [GEexiv2] This is a python wrapper to Exiv2, libraries to read and write metadata. Pyexiv2 (http://http://tilloy.net/dev/pyexiv2/overview.html) is now deprecated and the use of GExiv2 is now recommended. Installation of libexiv2 version 0.21 or newer and of Exiv2 (http://www.exiv2.org) are required. This libraries aren't part of standard Python installation but can be available also as separate package (try i.e: apt install python-pyexiv2 libexiv2-14). 


\end{description}
\end{description}

\section{Linux}
Everything was installed through distribution package manager. This list is not complete.

\begin{itemize}

\item Python2
\item Python2-numpy
\item OpenCV2
\item Exiv2
\item python2-exiv2
\item GExiv2
\item libgexiv2

\end{itemize}

\section{MacOS}

As an Unix-based operating system, installation of Python and its required modules is quite similar as on GNU/Linux. But some troubles can occur during installation of OpenCV2 and GExiv2 modules.

\subsection*{Packages manager}

Even if Python is natively installed in all Mac OS systems, it is recommended to install the lastest version of the branch 2.7 (i.e. Python 2.7.9). A safe way to install Python is to used a packages manager software like \textbf{Macports} or \textbf{Homebrew}. Here we will use the \textbf{Homebrew} software.

But first, we have to install \textbf{Apple Xcode} and \textbf{Apple Xcode Command Line Tools} from the AppStore. \textbf{Xcode} is an integrated development environment (IDE) to develop software on Apple systems. It contains the \textbf{GNU Compiler Collection} (GCC) required by \textbf{Macports} and \textbf{Homebrew} to compile source code of sofwares. So, go to the AppStore and install the software.

After installation of \textbf{Xcode}, we have to agree the license. Open the terminal and type the following command (you will be asked for your user password) :

\begin{verbatim}
$ sudo xcodebuild -license
\end{verbatim}

Then you have to download \textbf{Homebrew}. Follow the instructions on this page: http://brew.sh.

\subsection*{Python installation}

Installing Python with \textbf{Homebrew} is quite easy. Let's start by updating Homebrew repository.

\begin{verbatim}
$ brew update
\end{verbatim}

And let's proceed to the installation of Python.

\begin{verbatim}
$ brew install python
$ brew linkapps python
\end{verbatim}

The Python formulae also install \texttt{pip} and \texttt{Setuptools}. Both are tools for easily installing and managing Python packages. It is recommended to use \texttt{pip}. Let's install it

\begin{verbatim}
$ easy_install pip
\end{verbatim}

Now check the installed Python dependencies.

\begin{verbatim}
$ pip list
\end{verbatim}

To install a Python package:

\begin{verbatim}
pip install scipy
\end{verbatim}

That's all. Python and its basic modules are now installed. Let's check the installed version:

\begin{verbatim}
$ python -c 'import sys; print(".".join(map(str, sys.version_\
info[:3])))'
\end{verbatim}

\subsection*{OpenCV2 installation}

To install OpenCV2 module, follow these instructions:

\begin{verbatim}
$ brew tap homebrew/science
$ brew info opencv
$ brew install opencv
\end{verbatim}

Now, we have to link Python with these dependencies.

\begin{verbatim}
$ cd /Library/Python/2.7/site-packages
$ sudo ln -s /usr/local/Cellar/opencv/2.4.11/lib/python2.7/\
site-packages/cv.py cv.py
$ sudo ln -s /usr/local/Cellar/opencv/2.4.11/lib/python2.7/\
site-packages/cv2.so cv2.so
\end{verbatim}

\subsection*{GExiv2 installation}

\begin{verbatim}
$ brew install exiv2 pyexiv2
\end{verbatim}

I have a segmentation fault when importing...


\section{Windows}


\end{document}



